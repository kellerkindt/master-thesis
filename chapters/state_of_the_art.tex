\chapter{State of the art}

\section{Existing software solutions}

IBM InfoSpheere
\cite{infosphere:datastage}

GitLab \cite{gitlab:ci:yaml}

Jenkins \cite{jenkins:pipeline:jenkinsfile}

Quartz?? \cite{quartz:quickstart}

CSCS High Throughput Scheduler?? \cite{cscs:high_throughput}


qsub job submission % bad \url{https://wiki.uiowa.edu/display/hpcdocs/Basic+Job+Submission}



\subsection{Hadoop MapReduce}

focus transforming a big dataset by splitting it into many jobs, distributing it onto many workers, doing a transformation on each dataset, and merging it back together (only map -> reduce)
Distributed filesystem

\subsection{Quartz}

\cite{quartz:main}
\cite{quartz:overview}

\begin{itemize}
	\item + Java
	\item - requires integration
	\item - aimed towards running a job at a given time or in certain intervals
\end{itemize}

\subsection{Pipeline examples: Jenkins / GitLab}

Pipeline file with multiple stages
a stage can be executed on a host
focused on doing a job with different inputs again and again and again
CI -> usually no user interaction

\subsection{Camunda}

\cite{camunda:main}
\cite{camunda:process_engine_api}
\cite{camunda:rest_api_reference}

Rich Business Process Management tool, many types of tasks, steps, transitions, triggers and endpoints.
Focused upon moving a dataset along the matching path of the process.
Out of the box graphical user interface for process definition and interaction.
Allows custom external worker through queues.
Misses capability to control which task to process on which worker through fine grained filters and how to allocate and distribute resources(?).
Requires custom plugins for more advanced user forms, not designed for that.
Not designed provide an overview on the docker machines, cluster state nor logs, file up and download

\subsection{Cubernetes}

too heavy?

\subsection{Luigi}

Similar, but locked-down on python  (+machine learning)?
\cite{luigi:etc:distributed_pipelines}

\subsection{Celery}

\cite{celery:main}

\subsection{Nomad}

\todo{that dude that did the webui}

\cite{nomad:main}
Deployment and management of containers
rich REST API
can handle resource requirements
device plugins / GPU support

vs kubernetes \cite{nomad:vs:kubernetes}

++ available through debian / ubuntu std-repositories

\section{Docker}
