\chapter{State of the art}
\label{state_of_the_art}

In this chapter, programs solving similar problems, as described in the desired workflow, or dealing with a subset of the problem are looked into.
The reason for this is to use well established or suitable programs as middle-ware to reduce implementation overhead.
Where this is not possible, one might be able to gather ideas and learn about proven strategies to use or pitfalls to avoid while implementing custom solutions.

\section{Similar solutions}

This sections focuses on programs trying to provide somewhat similar workflows.

\subsection{Hadoop MapReduce}
\label{hadoop}
\label{mapreduce}

For big data transformation, Hadoop MapReduce\cite{hdfs:main} is well known.
With MapReduce, the input data is split into blocks of data and distributed onto Mappers.
Mappers then apply the business logic and output intermediate results in form of key/value pairs.
After shuffling, the Reduce stage will combine values from key/value pairs with the same key.
In the final output, each key is unique accompanied with a value.

This strategy has proven to be very powerful to process large amount input data because Mappers and Reducers can work independently on their data-sets and therefore scale very well when adding further instances.

If the implementation were to be based on Hadoop MapReduce to achieve the desired workflow, it could be done like the following:
\begin{itemize}
	\item Each video is split into many frames and each frame is applied to a Mapper
	\item A Mapper tries to detect all vehicles on a frame and outputs their position, orientation, size and so on
	\item The Reducer then tries to link the detections of a vehicle through multiple frames
	\item The final result would be a set of detections and therefore all positions for each vehicle in the video
\end{itemize}

But at the moment, this approach seems to be unfitting due to at least the following reasons:

\begin{enumerate}
	\item It is not always trivial to reasonable link the detections of a vehicle.
	For example, a vehicle can be hidden behind a tree for a few frames until visible again.
	In addition, MapReduce requires the combination to be performed per common key.
	Until one is trying to link the detections of multiple frames, there is no common identifier that could easily be used as key.
	The position of a moving vehicle cannot be used as key, neither can the color or size, because of the noise of the camera, deviation in detection output and perspective distortions.
	The current implementation of the TrackerApplication is archiving this by finding similarities between detections, but for the Mapper this would required to be expressed as a deterministic key.
	\item MapReduce is great in combining many machines to solve a big computational problem.
	But at the moment, this is neither a desired nor given condition. At the moment, there a handful of very powerful workstations with specialized hardware.
	Therefore it is perfectly acceptable and sometimes required, when each workstation works through a complete video at a given time instead.
\end{enumerate}

\subsection{Build Pipelines}

Build pipelines such as GitLab\cite{gitlab:main} and Jenkins\cite{jenkins:main} can also distribute the execution of a stage onto another server node.
In a common use-case, such build pipelines are used to build binaries out of source code, after a new commit into a SCM\footnote{Source Code Management} repository was made.
At IT-Designers GmbH GitLab as well as Jenkins are commonly used for scenarios exactly like this.
A pipeline definition in GitLab CI/CD \cite{gitlab:ci:yaml} or in a Jenkinsfile \cite{jenkins:pipeline:jenkinsfile} describe stages and commands to execute.
Each stage can be hosted on another node and be executed sequential or in parallel to each other.

Although this seems to be quite fitting for the desired workflow, there are two issues.
First of all, such a pipeline does not involve any user input besides an optional manual start command.
The result is then determined based on the state of the input repository.
Second, such an pipeline is designed to determine the output (usually by compiling) whereas each run ins independent from the previous and a repeated run shall provide the same result as the previous.
Usually, a new run is only caused by an change of the input data.
However, the desired workflow differs in this aspects.
A redo of a stage can depend on the result of the previous stage, for example if the results are poor or the the stage failed.
Instead of having multiple complete pipeline runs per project, the desired workflow uses a pipeline definition as base for which the order can be influenced.
Also, intermediate results need to influence further stages, even if repeated.


\subsection{Camunda}

Camunda\cite{camunda:main} calls itself a \enquote{Rich Business Process Management tool} and allows the user to easily create new pipelines by combining existing tasks with many triggers and custom transitions. % , many types of tasks, steps, transitions, triggers and endpoints.
Camunda is focused upon visualizing the flow and tracking the data through a pipeline. %moving a dataset along the matching path of the process.
The Camundas Process Engine\cite{camunda:process_engine_api} also allows user intervention between tasks.

One of the main supporting reason for it Camunda is the out of the box rich graphical user interface for process definition and interaction.
Through its API\cite{camunda:rest_api_reference}, Camunda also allows custom external workers to execute a task.
But it misses the capability to control which task shall be processed on which worker node which is required by the desired workflow.
It does also not provide any concept on how to allocate and distribute resources.
The user interface - while being rich overall - is quite rudimentary when it is about configuring tasks and would therefore require custom plugins to be developed for more advanced user interactions.

Camunda is also not designed to reorder stages or insert user interactions at seemingly random fashion.
Also, the user itself is considered more as a worker, that gets some request, \enquote{executes} this externally and finally marks the request as accepted or declined.
Mapping this to the desired workflow does not feel intuitive.
Finally, There is also no overview of task executors, no centralized log accumulation and no file up or download for project global resources.


\subsection{Nomad}

Nomad\cite{nomad:main} by HashiCorp is a tool to deploy, manage and monitor containers, whereas each job is executed in its own container.
It provides a rich REST API and can consider hardware constraints on job submissions.
Compared to Kubernetes\cite{nomad:vs:kubernetes}, which is similar but more focused on scaling containers to an externally applied load, it is very lightweight.
It is also available in many Linux software repositories - such as for Debian - which makes the installation very easy.

Because there were no grave disadvantages found (depending on a third party library can be a disadvantage in flexibility, error-pronous and limit functionality) Nomad is being considered as a middle-ware to manage and deploy stages.
Others\cite{nomad:etc:gui_thesis} seem to be using Nomad to manage and deploy containers for similar reasons.
Nonetheless, further testing and prototyping will be required for a final decision.


\subsection{dCache}

\enquote{The goal of this project is to provide a system for storing and retrieving huge amounts of data, distributed among a large number of heterogenous server nodes, under a single virtual filesystem tree with a variety of standard access methods}\cite{dcache:main}.
dCache seems to be able to solve the storage access and distribution for the stages and sever nodes.
When using dCache, once could store the global resources distributed on the server nodes.
Built-in replication would prevent access loss on a node or network failure and export through NFS\footnote{Network File System} allows easy access on Linux based systems\cite{dcache:overview:whitepaper}.

But the installation is complex and requires many services to be setup correctly, such as postgresql and many internal services such as zookeeper, admin, poolmanager, spacemanager, pnfsmanager, cleaner, gplazma, pinmanager, topo, info and nfs.
The documentation is also rather outdated and incomplete which meant, early tests with a prototype setup took days to setup and behaved rather unstable (probably due to a wrong configuration).
It is to be seen, whether such an complex and heavy system is actually required or if there are feasible alternatives.




\subsection{Further mentions}

The following list shall acknowledge programs that behave similar to the previously three mentioned strategies.
Programs there are listed here, were looked into, but not in-depth because while looking into them, miss-fits were detected early on. Listed in no specific order.

\begin{itemize}
	\item \textbf{Quartz}\cite{quartz:main} is a Java based program to schedule jobs. Instead of doing so by using input, Quartz executes programs through a timetable and in certain intervals.
	%	\cite{quartz:overview} 	\cite{quartz:quickstart}
	\item \textbf{Luigi}\cite{luigi:etc:distributed_pipelines} also executes pipelines with stages and is written in python. The advertised advantage is to define the pipeline directly in python code. But, this is at the same time the only way to define pipelines which contradictions with the existing Java TrackerApplication implementation.
	\item \textbf{Calery}\cite{celery:main} is focused on task execution through message passing and is written in Python. Intermediate results are expected to be transmitted through messages. Because is no storage strategy and python adapter-code would have been required, Calery was dismissed.
	\item \textbf{IBM InfoSphere}\cite{infosphere:datastage} provides similar to Camunda a rich graphical user interface but for data transfer. Dismissed due to commercial nature.
	\item \textbf{qsub}\cite{qsub:etc:wiki}\cite{qsub:etc:uiowa} CLI\footnote{Command Line Interface} using in HPC to submit jobs onto a cluster or grid. Dismissed due to an expected high setup overhead, non-required multi-user nature and the fact, that it only provides a solution for the job submission.
	\item \textbf{CSCS}\cite{cscs:high_throughput} High Throughput Scheduler (GREASY). Dismissed for similar reasons as qsub, although it is more light weighty and hardware agnostic (it can consider CUDA/GPU requirements).
	\item \textbf{zsync}\cite{zsync:main}, similar to rsync, is a file transfer program. Zsync allows to only transfer new parts when a file that shall be copied already exists in an older version on the target. This tool might be useful for a complete custom resource distribution strategy.
	\item \textbf{OpenIO}\cite{openio:main} provides a distributed file system, is already provided as Docker image and provides a simple to use CLI. Because the NFS export is only available through a paid subscription plan, it was dismissed from further investigation.
	\item \textbf{SeaweedFS}\cite{seeweedfs:main} provides a scalable and distributed file system. The most interesting aspects are that it is rack-aware as well as naively supports external storage such as Amazon S3. When adding server nodes from the cloud this could allow all nodes to access the same file system while using rack-aware replication to reduce used bandwidth and latency. A local test also proved that it is easy to setup, but because it cannot hot-swap nodes and was not able to recover when the seeweed master node became unreachable.
%	master halten die zuweisung file -> addresse
%	filer machen nur ein lookup und zuweisen oder sowas
%	aber der client frägt filer an
%	und der muss dann zu irgendeinem master die verbinundg aufbauen und nachschlagen
%	dh wollte pro physicalischen server 1 volume, 1 master, 1 filer haben
%	damit einfach dezentral kommen und gehen kann
%	aber problem 1: anzahl der master soll immer ungerade sein
%	problem 2: du kannst nicht einfach master on-the-fly hinzufügen und musst stattdessen teilweise die neu starten  mitm parameter: hey da drüben ist noch ein master
%	problem 3: es läuft nicht zuverlässig
	\item \textbf{Alluxio}\cite{alluxio:main} provides a distributed file system but was dismissed because it itself requires a centralized file system for the master and its fallback instances.
	\item \textbf{GlusterFS}\cite{glusterfs:main} is another tool to provide a distributed file system with replication. It was bought by IBM but is nonetheless available through the software repository of many Linux distributions such as Debian. A local test show that the setup is very easy without any configuration files required to be setup. The replication mechanism however, requires that an even multiple of nodes of the replica value are assigned to the file system. This makes GlusterFS hard to use in a scenario, where adding and removing nodes are expected to happen frequently, and therefore it was dismissed.
\end{itemize}

\section{Docker Integration}

As describe before (see \autoref{workflow}), for easy deployment, the implementation as well as the stages shall be executed inside Docker\cite{docker:main} containers.
This allows easier isolation of the stages and workspaces from each other and other host programs.
Because one needs to communicate with the Docker daemon, this increases the complexity for the implementation.
But thanks to third party libraries, the increase in complexity can be limited.


\todo{explain docker here already or only in the thesis?}


