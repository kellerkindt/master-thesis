\section{Evaluation}

The main focus of Winslow is to improve the efficiency in utilizing available hardware.
Without this automated approach, new stages were seldom started outside of work hours.
To estimate whether Winslow is providing any improvements, the actions that are scheduled by Winslow outside of these hours are to be assessed.

The following graphs are based on activity metrics, that were logged in the time period from the first usage in mid January 2020 until counter actions for COVID-19 in early March 2020 prevented active use due to the lock-down in Germany.
For each day of the week and hour of the day an associated cell shows the activity encoded in colours.
Furthermore, red lines show the begin and end of a typical work day.
Interestingly, the actual position of these red lines could also be inferred by the graphs themselves, due to activity spikes.


\begin{figure}[H]
	\centering
	\begin{tikzpicture}
		\begin{axis}[
			colormap/cool,
			colorbar,
			view={0}{90},
			xtick={1,3,5,7,9,11,13,15,17,19,21,23},
			ytick={1,2,3,4,5,6,7},yticklabels={Mo,Tu,We,Th,Fr,Sa,Su},
			ylabel={day of week},
			xlabel={hour of day},
			point meta max=0,
			point meta max=30,
			point meta=explicit,
			y dir=reverse,
			width=1.0\textwidth,
			height=0.5\textwidth]
			%			\addplot3[surf, shader=flat corner,mesh/cols=24,mesh/rows=7] table[x=hour, y=day, z=value, col sep=semicolon] {auto_starting.csv};
			\addplot[matrix plot*,mesh/cols=24,mesh/rows=7,shader=flat corner] table[x=hour,y=day,meta=value, col sep=semicolon] {active.csv};
			\addplot[mark=none, red] coordinates {(7.5,0) (7.5,8)};
			\addplot[mark=none, red] coordinates {(16.5,0) (16.5,8)};
		\end{axis}
	\end{tikzpicture}
	\caption{Actively running stages at a given day of the week and time of day}
	\label{metric:all_running}
\end{figure}

\autoref{metric:all_running} and \autoref{metric:all_started} show the overall activity.
In \autoref{metric:all_running} the number of running stages per hour is shown and in \autoref{metric:all_started} the number of started stages per hour is shown.
An increased activity between the read line - within the working hours - is visible.
But especially in \autoref{metric:all_started} an activity outside these times can be seen as well.

\begin{figure}[H]
	\centering
	\begin{tikzpicture}
		\begin{axis}[
			colormap/cool,
			colorbar,
			view={0}{90},
			xtick={1,3,5,7,9,11,13,15,17,19,21,23},
			ytick={1,2,3,4,5,6,7},yticklabels={Mo,Tu,We,Th,Fr,Sa,Su},
			ylabel={day of week},
			xlabel={hour of day},
			point meta max=0,
			point meta max=30,
			point meta=explicit,
			y dir=reverse,
			width=1.0\textwidth,
			height=0.5\textwidth]
			%			\addplot3[surf, shader=flat corner,mesh/cols=24,mesh/rows=7] table[x=hour, y=day, z=value, col sep=semicolon] {auto_starting.csv};
			\addplot[matrix plot*,mesh/cols=24,mesh/rows=7,shader=flat corner] table[x=hour,y=day,meta=value, col sep=semicolon] {starting.csv};
			\addplot[mark=none, red] coordinates {(7.5,0) (7.5,8)};
			\addplot[mark=none, red] coordinates {(16.5,0) (16.5,8)};
		\end{axis}
	\end{tikzpicture}
	\caption{Stages starting at a given day of the week and time of day}
	\label{metric:all_started}
\end{figure}

\autoref{metrics:auto_running} and \autoref{metrics:auto_started} show only stages that were automatically triggered by Winslow.




\begin{figure}[H]
	\centering
	\begin{tikzpicture}
		\begin{axis}[
			colormap/cool,
			colorbar,
			view={0}{90},
			xtick={1,3,5,7,9,11,13,15,17,19,21,23},
			ytick={1,2,3,4,5,6,7},yticklabels={Mo,Tu,We,Th,Fr,Sa,Su},
			ylabel={day of week},
			xlabel={hour of day},
			point meta max=0,
			point meta max=30,
			point meta=explicit,
			y dir=reverse,
			width=1.0\textwidth,
			height=0.5\textwidth]
			%			\addplot3[surf, shader=flat corner,mesh/cols=24,mesh/rows=7] table[x=hour, y=day, z=value, col sep=semicolon] {auto_starting.csv};
			\addplot[matrix plot*,mesh/cols=24,mesh/rows=7,shader=flat corner] table[x=hour,y=day,meta=value, col sep=semicolon] {auto_active.csv};
			\addplot[mark=none, red] coordinates {(7.5,0) (7.5,8)};
			\addplot[mark=none, red] coordinates {(16.5,0) (16.5,8)};
		\end{axis}
	\end{tikzpicture}
	\caption{Actively running stages at a given day of the week and time of day which were started automatically}
	\label{metrics:auto_running}
\end{figure}

\autoref{metric:all_running} clearly shows that there is quite a bit of activity even without direct user input.
This is especially obvious by looking at \autoref{metrics:auto_started}: there are many stages started in-between the late evening and midnight, a second wave at about one o'clock in the morning and activity on Saturdays.
Without Winslow these stages would not have been triggered.

The graphs imply that the system was actively used to schedule work to be executed on the weekends.
This can be seen in \autoref{metrics:auto_started} by the highlighted areas at late hours on Fridays, early and late hours on Saturdays as well as the continuous but decreasing overall activity from Saturday to Sunday in \autoref{metrics:auto_running}.
The gap of activity on Mondays at around noon in \autoref{metrics:auto_running} could indicate that the results were evaluated here.

\begin{figure}[H]
	\centering
	\begin{tikzpicture}
		\begin{axis}[
			colormap/cool,
			colorbar,
			view={0}{90},
			xtick={1,3,5,7,9,11,13,15,17,19,21,23},
			ytick={1,2,3,4,5,6,7},yticklabels={Mo,Tu,We,Th,Fr,Sa,Su},
			ylabel={day of week},
			xlabel={hour of day},
			point meta max=0,
			point meta max=30,
			point meta=explicit,
			y dir=reverse,
			width=1.0\textwidth,
			height=0.5\textwidth]
%			\addplot3[surf, shader=flat corner,mesh/cols=24,mesh/rows=7] table[x=hour, y=day, z=value, col sep=semicolon] {auto_starting.csv};
	       	\addplot[matrix plot*,mesh/cols=24,mesh/rows=7,shader=flat corner] table[x=hour,y=day,meta=value, col sep=semicolon] {auto_starting.csv};
	       	\addplot[mark=none, red] coordinates {(7.5,0) (7.5,8)};
	       	\addplot[mark=none, red] coordinates {(16.5,0) (16.5,8)};
		\end{axis}
	\end{tikzpicture}
	\caption{Stages automatically starting at a given day of the week and time of day}
	\label{metrics:auto_started}
\end{figure}


%\todo{to be added: table with more numbers, overall hours, hours in working time, hours outside working time, percentage of stages additionally executed thanks to Winslow, hours won by using Winslow...}

%\todo{table of number of stages executed, number of projects, manual trigger, auto trigger, average/min/max time?}
%\todo{percentage of stages triggered by user vs winslow}

For the graphs above, the following table summarizes the results:

\begin{table}[H]
	\centering
	\begin{tabular}{|l|r|r|r|}\hline
												& absolute 			& relative	& gain \\
		\hline
		All 		 							& 2003 h 			& 100 \% 	& \\
		Within work time						& 1575 h			& 78.6 \% 	& \\
		Outside work time	 					&  428 h			& 21.4 \% 	& \\
		Triggered by Winslow overall			& 1131 h			& 56.5 \%	& +129 \%\\
		Triggered by Winslow outside work time	&  286 h 			& 14.3 \% 	& +16 \%\\
\hline
	\end{tabular}
	\caption{Categorization of stage execution hours}
\end{table}

Overall, in the test period 2003 hours were spent computing - this includes overlapping stage executions - of which 78.6\% were executed within regular working hours.
To see the benefits of Winslow, the hours spent in stage executions that were automatically issued are of interest.
Here, Winslow triggered 1575 hours or 78.6\% of all stage executions which did not require staff assistance.
286 hours were overall gained by automatically triggering and performing stage executions outside of the working hours.
This is an increase of +16\% compared to no automated triggers.

To put this into perspective, typically week consists of 40 hours of work and 128 hours with the staff being absent.
So in theory 128 hours (+220\%) per week could additionally be spent executing stages, which raises the question why the numbers of Winslow are so small in comparison.

To answer this, one needs to look again at the heat-maps of \autoref{metric:all_running} and \autoref{metrics:auto_running}.
Winslow is being used, and as for \autoref{metrics:auto_started} discussed, it is being used for long running stage executions with one or two automatic triggers at nights.
But it is used even more during working hours, as can be seen in \autoref{metric:all_running}, where all pink cells fall within working hours.

In conclusion, Winslow is being used - and that actively, with over 2003 hours worth of compute time\footnote{in the previously mentioned time period}.
It is supporting the users by being the triggering for over half of all compute time spent and additionally enabling the staff to let large multi-stage tasks be executed over night or the weekend.
If it would not be beneficial to the staff, it would not be used in such an extend so shortly after implementation.

\begin{comment}
\todo{remove figure below?}
\begin{figure}[H]
	\centering
	\begin{tikzpicture}
		\begin{axis}[
			ybar interval,
  			xtick=,% reset from ybar interval
			width=1.0\textwidth,
			height=0.6\textwidth,
			xmin=0
			]
			\addplot+[hist={data=x}] file {duration.csv};
		\end{axis}
	\end{tikzpicture}
	\centering
	\begin{tikzpicture}
		\begin{axis}[
			width=1.0\textwidth,
			height=0.2\textwidth,
			boxplot/draw direction=x,
			xmin=0,xmax=280
		]
		\addplot+[boxplot={average=auto,data=x}] file {duration.csv};
		\end{axis}
	\end{tikzpicture}
	\caption{Stages automatically starting at a given day of the week and time of day}
\end{figure}
\end{comment}


%maybe with error scenario: being able to dynamically execute a stage of another project instead of stalling

%\todo{hours spent in work time} \todo{hours spent outside of work time} \todo{percentage win}
