\chapter{Conclusion}

Overall the project was a success.
Winslow helps to improve the usage of available hardware resources by additionally triggering stage executions in times where the staff is absent or not watching.
It helps in keeping order by separating workspaces and organizing them into projects.
The user interface provides an interaction method that is easy to understand for users, and provides upload and download mechanism without resorting back to console commands.

In the implementation phase an interesting and minimalistic way to synchronize events based on a commonly shared directory was discovered and implemented.
Winslow is easy to setup and due to its few dependencies and easy to maintain.
%Its architecture allows to be extended in further work.
The evaluation showed that Winslow is increasing productivity and hardware utilization without additional attention by the staff and that it is beneficial to the user.