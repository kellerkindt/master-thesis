
\chapter{System analysis}

\section{System context analysis}
System Umwelt Diagram


\begin{figure}[H]
	\begin{tikzpicture}[node distance=5,scale=0.8, every node/.style={transform shape}]
	\node [draw, rectangle, drop shadow, fill=white] (F) {Fahrzeug};
	\node [draw, circle, drop shadow, fill=white, text width=3cm, align=center] (C) [below right=of F] {MEC-View Server};
	\node [draw, rectangle, drop shadow, fill=white] (S) [below right=of C] {Sensor};
	%		\node [draw, rectangle, drop shadow, fill=white] (A) [above right=of C] {Fusionsalgorithmus};
	
	\node (ctrl1) [above=of S] {};
	\node (ctrl2) [above right=of C] {};
	\node (ctrl3) [right=of C] {};
	\node (ctrl4) [below=of F] {};
	\node (ctrl5) [left=of C] {};
	\node (ctrl6) [below=of ctrl5] {};
	\node (ctrl7) [above=of C] {};
	\node (ctrl8) [right=of F] {};
	
	%
	% Fahrzeug -> MEC-View-Server
	%
	\path[draw, -{Latex[scale=2.0]}, dashed] (F)
	edge [bend right] node [fill=white] {deabonnieren} (C)
	edge [bend left]  node [fill=white] {abonnieren} (C)
	.. controls (ctrl7) and (ctrl8) ..  node [fill=white] {Als Fahrzeug registrieren} (C);
	
	%
	% MEC-View-Server -> Fahrzeug
	%
	\path[draw, -{Latex[scale=2.0]}] (C)
	edge node [fill=white] {EnvironmentFrame} (F)
	.. controls (ctrl4) and (ctrl5) ..  node [fill=white] {Sektoren} (F);
	
	%
	% MEC-View-Server -> Algorithmus
	%
	%		\path[draw, -{Latex[scale=2.0]}] (C)
	%				edge [bend right] node [fill=white] {SensorFrame} (A);
	
	%
	% Algorithmus -> MEC-View-Server
	%
	%		\path[draw, -{Latex[scale=2.0]}] (A)
	%				edge [bend right] node [fill=white] {EnvironmentFrame} (C);
	
	%
	% Sensor -> MEC-View-Server
	%		
	\path[draw, -{Latex[scale=2.0]}, dashed] (S)
	.. controls (ctrl3) and (ctrl1) ..  node [fill=white] {Als Sensor registrieren} (C);
	
	\path[draw, -{Latex[scale=2.0]}] (S)
	edge node [fill=white] {Sensor(Idle)Frame} (C);
	
	%
	% MEC-View-Server -> Sensor
	%
	\path[draw, -{Latex[scale=2.0]}, dashed] (C)
	edge [bend left]  node [fill=white] {abonnieren} (S)
	edge [bend right] node [fill=white] {deabonnieren} (S);
	
	
	
	
	\path[draw, -{Latex[scale=2.0]}] (4, -8) -- (.5, -8) node [pos=.5, above] {Datenfluss};
	\path[draw, -{Latex[scale=2.0]}, dashed] (4, -9) -- (.5, -9) node [pos=.5, above] {Kontrollfluss};
	
	%\umlinherit{F}{L}
	%\umlinherit{S}{L}
	\end{tikzpicture}
	\centering
	\caption{\todo{Systemkontextdiagramm}}
	\label{system_context_diagramm}
\end{figure}


System Interacts with the User and other instances of itself


\section{Use Case Diagrams}

Finding all requirements can be challenging.
Drawing Use Case Diagrams can help to discover requirements while being very easy to understand.
This helps in understanding the customers needs \cite{Rosenberg2007} while the customer receives an impression on what will be reflected in the final product.

Listing all use cases in a single diagram negates the desired effect of it being easily understandable because of the number of interaction possibilities.
Instead, the interactions are grouped in categories and displayed in separate diagrams.
\autoref{use_case:overview} shows high-level use cases of all categories that are relevant to a user of the system.

\begin{figure}[H]
	\scalebox{.65}{
		\begin{tikzpicture}[node distance=5]
		\begin{umlsystem}{Winslow - Interaction Categories}
		\umlusecase[y=0,name=u1]{Manage Pipelines and Projects [\autoref{use_case:mgmt}]}
		\umlusecase[y=-1,name=u2]{Manage Resources and Workspaces [\autoref{use_case:files}]}
		\umlusecase[y=-2,name=u3]{Manage and Monitor Execution [\autoref{use_case:execution}]}
		\umlusecase[y=-3,name=u4]{Monitor Nodes [\autoref{use_case:monitor}]}
		\end{umlsystem}
		\umlactor[x=-9.5,y=-1.5]{User}
		\umlassoc{User}{u1}
		\umlassoc{User}{u2}
		\umlassoc{User}{u3}
		\umlassoc{User}{u4}
		\end{tikzpicture}
	}
	\centering
	\caption{Use Case Diagram showing the top level interaction categories}
	\label{use_case:overview}
\end{figure}

\subsection{Managing Pipelines and Projects}
\label{use_case:mgmt}

\todo{.}

\begin{figure}[H]
	\scalebox{.65}{
		\begin{tikzpicture}[node distance=5]
		\begin{umlsystem}{Winslow - Manage Pipelines and Projects}
		\umlusecase[y=0,name=u110]{\reqNameRef{mgmt:create:pipeline}}
		\umlusecase[y=-1,name=u120]{\reqNameRef{mgmt:update:pipeline}}
		\umlusecase[y=-2,name=u130]{\reqNameRef{mgmt:delete:pipeline}}
		\umlusecase[y=-3,name=u140]{\reqNameRef{mgmt:list:pipeline}}
		
		\umlusecase[y=-5,name=u210]{\reqNameRef{mgmt:create:project}}
		\umlusecase[y=-6,name=u220]{\reqNameRef{mgmt:update:project:pipeline}}
		\umlusecase[y=-7,name=u230]{\reqNameRef{mgmt:update:project:name}}
		\umlusecase[y=-8,name=u240]{\reqNameRef{mgmt:update:project:labels}}
		\umlusecase[y=-9,name=u250]{\reqNameRef{mgmt:delete:project}}
		\umlusecase[y=-10,name=u260]{\reqNameRef{mgmt:list:project}}
		\end{umlsystem}
		\umlactor[x=-9.5,y=-4]{User}
		\umlassoc{User}{u110}
		\umlassoc{User}{u120}
		\umlassoc{User}{u130}
		\umlassoc{User}{u140}
		\umlassoc{User}{u210}
		\umlassoc{User}{u220}
		\umlassoc{User}{u230}
		\umlassoc{User}{u240}
		\umlassoc{User}{u250}
		\umlassoc{User}{u260}
		\end{tikzpicture}
	}
	\centering
	\caption{Use Case Diagram showing the general management interactions}
\end{figure}


\subsection{Managing Resources and Workspaces}
\label{use_case:files}

\todo{.}

\begin{figure}[H]
	\scalebox{.65}{
		\begin{tikzpicture}[node distance=5]
		\begin{umlsystem}{Winslow - Manage Resources and Workspaces}
		\umlusecase[y=0,name=u110]{\reqNameRef{files:upload}}
		\umlusecase[y=-1,name=u120]{\reqNameRef{files:download}}
		\umlusecase[y=-2,name=u130]{\reqNameRef{files:list}}
		\end{umlsystem}
		\umlactor[x=-9.5,y=-1]{User}
		\umlassoc{User}{u110}
		\umlassoc{User}{u120}
		\umlassoc{User}{u130}
		\end{tikzpicture}
	}
	\centering
	\caption{Use Case Diagram showing the general management interactions}
\end{figure}

\subsection{Managing and Monitoring Executions}
\label{use_case:execution}

\todo{.}

\begin{figure}[H]
	\scalebox{.65}{
		\begin{tikzpicture}[node distance=5]
		\begin{umlsystem}{Winslow - Manage and Monitor Executions}
		\umlusecase[y=0,name=u110]{\reqNameRef{exec:start:stage}}
		\umlusecase[y=-1,name=u120]{\reqNameRef{exec:pause:stage}}
		\umlusecase[y=-2,name=u130]{\reqNameRef{exec:resume:stage}}
		\umlusecase[y=-3,name=u140]{\reqNameRef{exec:abort:stage}}
		\umlusecase[y=-5,name=u150]{\reqNameRef{exec:inspect:logs}}
		\umlusecase[y=-6,name=u160]{\reqNameRef{exec:inspect:state}}
		\end{umlsystem}
		\umlactor[x=-9.5,y=-3]{User}
		\umlassoc{User}{u110}
		\umlassoc{User}{u120}
		\umlassoc{User}{u130}
		\umlassoc{User}{u140}
		\umlassoc{User}{u150}
		\umlassoc{User}{u160}
		\end{tikzpicture}
	}
	\centering
	\caption{Use Case Diagram showing the general management interactions}
\end{figure}

\subsection{Monitoring Nodes}
\label{use_case:monitor}
\todo{.}
\begin{figure}[H]
	\scalebox{.65}{
		\begin{tikzpicture}[node distance=5]
		\begin{umlsystem}{Winslow - Monitor Nodes}
		\umlusecase[y=0,name=u110]{\reqNameRef{node:monitor:cpu}}
		\umlusecase[y=-1,name=u120]{\reqNameRef{node:monitor:ram}}
		\umlusecase[y=-2,name=u130]{\reqNameRef{node:monitor:netio}}
		\end{umlsystem}
		\umlactor[x=-9.5,y=-1]{User}
		\umlassoc{User}{u110}
		\umlassoc{User}{u120}
		\umlassoc{User}{u130}
		\end{tikzpicture}
	}
	\centering
	\caption{Use Case Diagram showing the general management interactions}
\end{figure}


\subsection{System Administration}

Furthermore to the interactions with a user, the system must provide further capabilities that are \todo{of} concern to the administrator.


\begin{figure}[H]
	\scalebox{.65}{
		\begin{tikzpicture}[node distance=5]
		\begin{umlsystem}{Winslow - Administration}
		\umlusecase[y=0,name=u1]{Start and Add a new Node}
		\umlusecase[y=-2,name=u2]{Stop and Remove a Node}
		\umlusecase[y=-4,name=u3]{Debug behavior of a Node}
		\end{umlsystem}
		\umlactor[x=-9.5,y=-2]{Admin}
		\umlassoc{Admin}{u1}
		\umlassoc{Admin}{u2}
		\umlassoc{Admin}{u3}
		\end{tikzpicture}
	}
	\centering
	\caption{Use Case Diagram showing administrative interactions}
	\label{use_case:admin}
\end{figure}

\section{Communication / message analysis ?}

\section{Interface analysis}