\chapter{The Program}

This chapter will discuss the program which shall be implemented.
To do so, first, the problem to solve must be understood.
Furthermore, existing solutions have to be considered for being used or being used as middle-ware or as being positive or negative examples.

\section{(Ideal) Workflow}

To understand the problem that shall be solved, the \todo{current workflow is considered} to be able to imagine and ideal workflow.
The program shall then enable the ideal workflow.

\todo{write more like the user first does abc then wants to do def...?}

A tracking project starts with a video as input.
The video is usually taken in 4k (3840 x 2160 pixels), a \todo{8 bit RGB} color depth and with 24 frames per second.
A raw frame is therefore least 189 MiB large.
The camera drone is using H.264 to compress the video to about \todo{X GiB} per minute.
The drone can fly for 25 Minutes straight.
A single video can therefore be as large as \todo{X GiB}.
The system must be able to reliable store and access such an video.
The user wants to upload such an video to the system in a common and easy way.

A user then wants to select a pre-defined pipeline to be executed for the uploaded video.
In this context, a pipeline shall describe the general order of compute steps, called stages while the combination of a pipeline and a video shall be called project.
Each stage applies an operation onto the whole video, such as detecting cars, tracking stationary reference points, estimating the camera position and assigning trajectories onto detections.
The logic of such an stage is provided and not within the scope of the program to realize, but needs to be invoked and monitored instead.
After selecting a pipeline, the user might want to set additionally environment variables or upload prepared files into the workspace.

The user wants to be informed about the current progress of all projects, a certain pipeline or stage.
Therefore a current state is of interest, such as running, failed or succeeded for the pipeline and each stage.

Any stage might require additional user input, such as manually confirming that tracked references points are indeed reasonable.
To do so, each stage must be able to pause the pipeline and the user must be able to access the intermediate results of a stage.
This means, the user must able to browse, download and upload files.
A stage might also have additional requirements or preferences for certain hardware, such as a GPU for the neuronal network.
Which stage is executed on which execution node, is assigned by the system and not of special interest to the user. \todo{wait for node to be free?}

The system also needs to provide the current console output of a stage, so that the user can pause a pipeline manually if needed.
In contrasted to pausing, the user must also be able to resume a pipeline.

The user might find the results of a stage not meeting the expectations.
In such a scenario, the users wants to redo the stage with adjusted input parameters such as environment variables, or a differently prepared workspace.

\todo{scalability, it shall be easy to add and remove hardware-nodes}