

\chapter{Introducing Winslow}

\begin{wrapfigure}{r}{0.37\textwidth}
	\centering
	\vspace{-1.5cm}
	\includegraphics[width=0.3\textwidth]{winslow_friendly_flipped.png}
	\label{winslow:icon}
	\caption{Winslow Logo}
\end{wrapfigure}


The program that shall implement the requested features listed by \autoref{requested-features} is being called Winslow.
This name refers to Frederick Winslow Taylor who was an American mechanical engineer and one of the first management consultants in the 19th century\cite[wiki:winslow].
Both strive to make people's work more efficient.


\section{Common Terminology}
\label{winslow:terminology}

This chapter explains the meaning of words and terminology used later on.
It is crucial that every reader has the same understanding for the words used so that there are no wrong assumptions, expectations or surprises.

The root of a work item is called project.
A project will usually refer to one video footage that shall be processed.
Each project has its exclusive directory for input, output and intermediate files, called workspace.
Furthermore a project as an author and possibly participants, that can help in steering and monitoring the processing.
To do so, there is always a pipeline assigned to a project.

A pipeline consists of at least one but usually multiple stages that can at least be processed linearly - one after the other.
Furthermore, a pipeline can define environment variables that valid for all stages within the stage.

A stage is the smallest work unit that can be executed.
A docker image\autoref{docker:image} is specified as execution environment as well as further stage specific environment variables, command line arguments and hardware requirements (such as CPU cores, GPUs and minimum available RAM).
This enables a stage to use a common image as well as to specify very precisely how to process the data.

In addition, stages and pipelines can be distinguished in a active, running or completed stage or pipeline and a definition.
A stage definition specifies the above mentioned presets - allowing the user to adjust details just before submission - to create multiple stages from, while a running or complete stage has all information of what is or was executed and allows no further alteration.
For pipeline definitions this is similar, a pipeline definition can be used to specify a common order of execution.
Once assigned to a project, the project can freely adjust and change its instance of the pipeline.

A running process that is executing the binary of Winslow is called Winslow instance.

A computer with a Winslow instance that is accepting or executing stages is called execution node.
The focus relies here on the computer providing compute resources to Winslow.

